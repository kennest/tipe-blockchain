% !TEX TS-program = pdflatex
% !TEX encoding = UTF-8 Unicode


\documentclass[11pt,a4paper]{article}

%\usepackage[francais]{babel}
\usepackage[T1]{fontenc}
\usepackage[utf8]{inputenc}

\usepackage{amsmath,amsfonts,amssymb}

\usepackage{geometry}
\geometry{margin=75pt}

\usepackage[upright]{fourier}
\usepackage{subfig}

\usepackage{shadethm}

\usepackage{color}
\definecolor{gris_clair}{gray}{.9}
\definecolor{gris}{gray}{.35}
\definecolor{vert}{rgb}{0,0.5,0}
\definecolor{rouge}{rgb}{0.5,0,0}
\definecolor{turquoise}{rgb}{0,0.5,0.5}

\usepackage{listings}
\usepackage{paralist}
\usepackage{stmaryrd}
\usepackage{tikz}
\usetikzlibrary{shapes.multipart}
\usetikzlibrary{calc}

\lstset{
language=Python,
backgroundcolor=\color{gris_clair},
frame=single,
basicstyle=\footnotesize\ttfamily\color{gris},
identifierstyle=\color{black},
keywordstyle=\color{vert},
stringstyle=\color{rouge}, showstringspaces=false,
commentstyle=\itshape\color{turquoise},
%numbers=left, numbersep=5pt, numberstyle=\color{gris}\tiny,stepnumber=5,
breaklines=true,
literate=
  {é}{{\'e}}1 {É}{{\'E}}1 {à}{{\`a}}1 {è}{{\`e}}1% 
  {À}{{\`A}}1 {È}{{\'E}}1 {ë}{{\"e}}1 {ï}{{\"i}}1%
  {â}{{\^a}}1 {ê}{{\^e}}1 {î}{{\^i}}1 {ô}{{\^o}}1% 
  {û}{{\^u}}1 {Â}{{\^A}}1 {Ê}{{\^E}}1 {Î}{{\^I}}1%
  {Ô}{{\^O}}1 {œ}{{\oe}}1 {Œ}{{\OE}}1 {æ}{{\ae}}1%
  {Æ}{{\AE}}1 {ç}{{\c c}}1 {Ç}{{\c C}}1 {€}{{\EUR}}1 ,
morekeywords={len,input,range}}         


\title{Mise en cohérence des objectifs du TIPE}
%
%
% Propagation des informations dans un réseau pair-à-pair
%
%

\author{Dimitri Granger \and Matthias Goffette}

\begin{document}

\newshadetheorem{defin}{Définition}
\newshadetheorem{theo}{Théorème}

\maketitle


\section{Positionnement thématique}


% Uniquement les gros titres ou les plus petits items ?
%
% 3 en tout !

\emph{Informatique pratique, Informatique théorique, Mathématiques - Autres Domaines}


\section{Mots-clefs}

%
% Une colonne en français, une autre en anglais
%

\begin{it}
Graphe
Système multi-agent
Réseau robuste
Connectivité
Transmission de l'information
\end{it}


\section{Bibliographie commentée}

		Il semble qu'un réseau en étoile, c'est-à-dire ressemblant à l'architecture client-serveur, permet au \emph{Designer} de protéger plus facilement le réseau\cite{goyal14}. Cet article utilise des probabilités pour mesurer qui est le vainqueur. Il semble intéressant de reproduire cette expérience, et en modifier les paramètres, de manière à observer des différences en terme de stratégie optimale. En effet, il nous semblait, avant de lire cet article, qu'un réseau en forme de \emph{graphe complet} (chaque noeud est lié à tous les autres) aurait été le meilleur. Ceci dépend vraisemblablement du but de l'attaquant (corrompre un noeud précis, s'emparer de la totalité du réseau...) et des paramètres utilisés pour la modélisation.




\section{Problématique retenue}

Plusieurs problématiques s'offrent à nous :
\begin{itemize}
	\item Quelle architecture du réseau permet la meilleure résistance aux attaques, tout en gardant de bonnes vitesses de propagation des données ?
	\item \textbf{Comment rechercher efficacement une information sur un réseau ? Application : Moteur de recherche décentralisé.}
	\begin{itemize}
		\item Comment s'assurer que les informations reçues sont \emph{de confiance}, c'est-à-dire vraie et correspondant à une réponse à ce qui a été demandé ?
	\end{itemize}
	
	\item (Comment organiser le partage de fichiers efficacement (cas d'un iso \texttt{linux}) ? Rajouter des contraintes : combien de noeuds maximum ?)
	
\end{itemize}

\section{Objectifs du TIPE}

Notre but consiste à créer un réseau permettant une transmission des informations rapides, tout en résistant efficacement aux attaques, lors desquelles un attaquant prendrait possession de plusieurs n{\oe}uds.



%	\begin{thebibliography}{56}
%	
%%	\bibitem{lamport94}
%%	  Leslie Lamport,
%%	  \emph{\LaTeX: a document preparation system},
%%	  Addison Wesley, Massachusetts,
%%	  2nd edition,
%%	  1994.
%	  
%	\bibitem{goyal14}
%	 S. Goyal, A. Vigier
%	 \emph{Attack, Defense and Contagion in Networks}
%	 2014-01-18
%	
%	\end{thebibliography}


\bibliographystyle{plain}
\bibliography{references} 

\end{document}