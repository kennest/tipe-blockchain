% !TEX TS-program = pdflatex
% !TEX encoding = UTF-8 Unicode


\documentclass[12pt,a4paper]{article}

%\usepackage[french]{babel}
\usepackage[T1]{fontenc}
\usepackage[utf8]{inputenc}

\usepackage{amsmath,amsfonts,amssymb}

\usepackage{geometry}
\geometry{margin=75pt}

\usepackage[upright]{fourier}
\usepackage{subfig}

\usepackage{shadethm}

\usepackage{color}
\definecolor{gris_clair}{gray}{.9}
\definecolor{gris}{gray}{.35}
\definecolor{vert}{rgb}{0,0.5,0}
\definecolor{rouge}{rgb}{0.5,0,0}
\definecolor{turquoise}{rgb}{0,0.5,0.5}

\usepackage{listings}
\usepackage{paralist}
\usepackage{stmaryrd}
\usepackage{tikz}
\usetikzlibrary{shapes.multipart}
\usetikzlibrary{calc}

\lstset{
language=Python,
backgroundcolor=\color{gris_clair},
frame=single,
basicstyle=\footnotesize\ttfamily\color{gris},
identifierstyle=\color{black},
keywordstyle=\color{vert},
stringstyle=\color{rouge}, showstringspaces=false,
commentstyle=\itshape\color{turquoise},
%numbers=left, numbersep=5pt, numberstyle=\color{gris}\tiny,stepnumber=5,
breaklines=true,
literate=
  {é}{{\'e}}1 {É}{{\'E}}1 {à}{{\`a}}1 {è}{{\`e}}1% 
  {À}{{\`A}}1 {È}{{\'E}}1 {ë}{{\"e}}1 {ï}{{\"i}}1%
  {â}{{\^a}}1 {ê}{{\^e}}1 {î}{{\^i}}1 {ô}{{\^o}}1% 
  {û}{{\^u}}1 {Â}{{\^A}}1 {Ê}{{\^E}}1 {Î}{{\^I}}1%
  {Ô}{{\^O}}1 {œ}{{\oe}}1 {Œ}{{\OE}}1 {æ}{{\ae}}1%
  {Æ}{{\AE}}1 {ç}{{\c c}}1 {Ç}{{\c C}}1 {€}{{\EUR}}1 ,
morekeywords={len,input,range}}         


\title{Comment optimiser l'architecture d'un réseau pour résister aux attaques ?\\
  Rapport final de TIPE}
%
% Propagation des informations dans un réseau pair-à-pair
%
%

\author{Matthias Goffette}

\begin{document}
\maketitle

\section{Abstract} % 100 mots | 109 mots

	Networks need to be efficient, in terms of communication speed, and reliable, so that they can resist to accidents and attacks. We focus on two types of networks : homogenous and scale-free. We model the networks as a graph, each node representing an agent. On these, attacks are performed. The methodology is the following : an agent has an initial true information. Then, it passes it to its neighbours. If they are normal agents, they repeat the process, but if they are attackers, they falsify the information before spreading them. The simulation show that for the same mean degree of nodes, homogenous networks are more resiliant than scale-free networks.



\section{Préambule} % 75 mots | 55 mots

	Mon objectif a peu changé depuis la MCOT. J'ai donc poursuivi l'étude de l'impact d'une attaque sur les noeuds du réseau, selon sa topologie. Cependant, je ne me suis finalement pas concentré sur la vitesse de transmission des informations. En effet, cette partie m'a semblé moins intéressante. J'ai préféré me concentrer sur l'étude des attaques.


\section{Introduction} % 100 mots | 96 mots
% Le  candidat  doit  introduire  de  façon  claire  et  synthétique  le travail  qu’il  a effectivement réalisé, en cohérence avec son préambule.


Sécuriser les réseaux informatique est aujourd'hui un point primordial.
En effet, ils sont le support de communications de plus en plus nombreuses entre les ordinateurs.
Dans ce cadre, je me suis intéressé à l'effet de l'architecture d'un réseau sur sa vulnérabilité aux attaques.
Pour apporter des éléments de réponse à cette question, j'ai utilisé une modélisation multi-agents en langage Python.
J'ai effectué des simulations sur deux types de réseaux, \emph{scale-free} et \emph{homogène}.
J'ai fait varier deux paramètres de la taille du réseau : le nombre de nœuds et le nombre d'arêtes par nœud.  


\section{Corps Principal} %750 mots | 411 mots

\subsection{Modalités d'action} % 250 | 182 mots

Il a fallu tout d'abord définir la manière dont une attaque allait opérer.
Je me suis penché sur une attaque sur les noeuds.
C'est là que réside la différence avec le travail de Dimitri Granger, l'autre membre du groupe, qui s'est intéressé à des attaques sur les liens.
Ainsi, un noeud peut être soit normal, soit attaquant.
Un noeud normal transmettra à ses voisins les informations qu'il reçoit sans les modifier.
Mais un attaquant, avant de transmettre des informations, les faussera.


J'ai ensuite choisi la représentation des objets utiles pour la modélisation d'attaques sur un réseau.
Pour cela, j'ai utilisé des objets Pythons : les agents qui représentent les noeuds du réseau, les tunnels qui représentent les arêtes, et un objet qui rassemble les deux premiers, le réseau.
Un dernier objet, l'information, est utilisé pour observer la propagation d'informations faussées par les attaquants.
Une information peut prendre deux valeurs, vrai ou faux.


J'ai concentré mon étude sur les réseaux homogènes, et \emph{scale-free}.
Pour générer des réseaux \emph{scale-free}, j'ai utilisé l'algorithme de Barabási–Albert.
Il consiste à prendre un graphe initial, et à ajouter des nœuds.
A chaque ajout d'un nœud $i$, on le lie à un nœud $j$ avec une probabilité proportionnelle à la connectivité de $j$.
Cela crée un réseau dans lequel la probabilité qu'un noeud ait $k$ voisins est $\alpha k^{-\gamma}$.
Ici, $\gamma = 3$ et $\alpha \approx 0.83$.
Ainsi, quelques nœuds ont une forte connectivité, et une majorité en ayant une faible.
Pour générer des réseaux homogènes, j'ai utilisé la bibliothèque Python NetworkX.

Ensuite, j'ai défini les expériences que je souhaitais réaliser.
Puis j'ai conçu l'architecture du code me permettant de réaliser ce plan d'expérience.


\subsection{Restitution des résultats} %300 | 376

Dans le cas des réseaux homogènes, j'ai fait varier le nombre de nœuds, j'ai considéré des réseaux ayant 10, 50 puis 100 nœuds, avec dans chaque cas trois arêtes par nœud.
J'ai également fait varier le nombre d'arêtes par nœud, pour un réseau de 50 agents, avec des degrés de 3 et 20 successivement.

Pour chaque jeu de paramètres, j'ai effectué 100 simulations.
Une simulation consiste en la génération d'un réseau.
Pour chacun de ces réseaux, j'ai fait varier le nombre d'attaquants entre 0 et le nombre de nœuds $n$ du réseau.
Pour chacun de ces cas, l'algorithme a effectué une diffusion de l'information dans le réseau.
Il en résulte une proportion d'informations fausses en fonction du nombre d'attaquants.

La comparaison de la vulnérabilité des différents réseaux en fonction de leur architecture se fonde sur la moyenne des 100 simulations. 


Sur les réseaux homogènes, conformément à mes attentes, j'observe que le nombre d'informations fausses dans le réseau croît avec le nombre d'attaquants (voir figure \ref{noeuds}).
La courbe est concave : plus il y a d’attaquants, moins l’action d’en ajouter un nouveau a un effet important.
Cela s'explique par une redondance d'informations fausses.
Pour un degré constant, la diffusion d'informations fausses est plus importante pour un réseau de grande taille.

\begin{center}
\includegraphics[width=0.49\linewidth]{../resultats/atkaleat/atkaleat-50-3-2-i100.png}
\includegraphics[width=0.49\linewidth]{../resultats/atkaleat/atkaleat-100-3-1-i100.png}
\captionof{figure}{Réseau homogène - Variation du nombre de nœuds} \label{noeuds}
\end{center}

Pour un nombre de nœuds constant, l'augmentation du nombre d'arêtes par nœuds se traduit, au-delà d'un certain seuil, par l'apparition d'une partie affine de la courbe, qui correspond à un stade où les seuls nœuds non attaquants sont voisins de l'émetteur (voir figure \ref{aretes}).

\begin{center}
\includegraphics[width=0.49\linewidth]{../resultats/atkaleat/atkaleat-50-3-2-i100.png}
\includegraphics[width=0.49\linewidth]{../resultats/atkaleat/atkaleat-50-20-2-i100.png}
\captionof{figure}{Réseau homogène - Variation du nombre d'arêtes par nœud} \label{aretes}
\end{center}

Dans le cas des réseaux invariants d'échelle, on observe en moyenne le même type de courbe concave que pour un réseau homogène.
Cependant, pour une simulation donnée, on voit qu'il s'agit d'une courbe à paliers (voir figure \ref{sf}).
En effet, lorsqu'un nœud ayant une forte connectivité devient attaquant, il a un fort impact sur le reste du réseau, d'où l'apparition de paliers de diffusion.

\begin{center}
\includegraphics[width=0.49\linewidth]{../resultats/scale-free/sf-t1s-n100-2-it3.png}
\includegraphics[width=0.49\linewidth]{../resultats/scale-free/sf-t1-n100-1-it100.png}
\captionof{figure}{Réseaux scale-free - Courbes de trois simulations et moyenne sur cent simulations} \label{sf}
\end{center}

Pour une même somme de degrés des nœuds attaquants, qui traduit l'influence des attaquants, le nombre d'attaquants varie beaucoup (voir figure \ref{deg}).


\begin{center}
\includegraphics[width=0.49\linewidth]{../resultats/scale-free/sf-t2-n-100-1-it100.png}
\captionof{figure}{Etude de la somme des degrés attaquants selon le nombre d'attaquants} \label{deg}
\end{center}

Pour comparer les deux types de réseaux, j'ai entrepris d'établir le degré moyen d'un nœud dans un réseau invariant d'échelle.
Ce degré moyen vaut deux.
Alors, pour un même degré moyen, on constate que les réseaux homogènes ont en moyenne une plus grande proportion d'informations fausses (voir figure \ref{comp}).


\begin{center}
\includegraphics[width=0.49\linewidth]{../resultats/atkaleat/atkaleat-100-2-1.png}
\includegraphics[width=0.49\linewidth]{../resultats/scale-free/sf-t1-n100-1-it100.png}
\captionof{figure}{Comparaison des réseaux homogène et invariant d'échelle} \label{comp}
\end{center}
	



\subsection{Analyse - Exploitation - Discussion} %200 | 75

Si les résultats sur les réseaux homogènes peuvent être intéressants, ce type de réseau est peu facile à mettre en pratique.
C'est un objet principalement théorique.
De plus, on a supposé que la probabilité d'être attaquant ne dépend pas de la connectivité.
Or en pratique, les serveurs très connectés sont de gros serveurs, généralement plus protégé que les autres.
Ils devraient donc être moins souvent infecté, ce qui augmente les performance des réseaux invariants d'échelle.


\section{Conclusion générale} %75 mots | 33 mots

	Les résultats valident les hypothèses que nous avions formulées. Mais on voit que les réseaux homogènes, bien que plus difficiles à mettre en place, semblent être moins vulnérables que les réseaux scale-free. 
Le modèle des réseaux \emph{scale-free} a une importance pratique, puisque beaucoup de réseaux réels, comme Internet, prennent cette forme. Les réseaux homogènes sont plus difficiles à mettre en pratique, et donc peu utilisés. Mais il peremettent une base intéressante de comparaison.



%
%
%
%% Uniquement les gros titres ou les plus petits items ?
%%
%% 3 en tout !
%
%\emph{Informatique pratique, Informatique théorique, Mathématiques - Autres Domaines}
%
%
%\section{Mots-clefs}
%
%%
%% Une colonne en français, une autre en anglais
%%
%
%\begin{it}
%\begin{itemize}
%	\item Graphe | Graph
%	\item Système multi-agent | Agent-based systm
%	\item Réseau robuste | Robust network
%	\item Connectivité | Connectivity
%	\item Transmission de l'information | Data transmission
%\end{itemize}
%\end{it}
%
%
%\section{Bibliographie commentée}
%
%
%
%
%	Les réseaux, qu'ils soient physiques ou informatiques, sont vulnérables aux des attaques, qu'elles soient intelligentes ou non. Il convient donc de chercher comment protéger les protéger de telles attaques. Les réseaux permettant la circulation d'informations et de biens, le but de la défense est de garantir, dans un réseau informatique, la véracité des informations qui circulent, et, de manière générale, la connexité du réseau, pour qu'il reste possible de le parcourir. Notre modélisation retient deux sortes d'attaques : la première étant celle d'un utilisateur malveillant qui chercherait à prendre le contrôle d'un réseau informatique pour répandre de fausses informations, la seconde étant la suppression d'arêtes ou de noeuds composant le réseau. La structure même d'un réseau peut être mise en danger, par exemple par une catastrophe naturelle qui détruirait les câbles ou les centrales électriques. La question est alors de trouver comment organiser un réseau pour qu'il reste fonctionnel, c'est à dire connexe, même après la destruction de certains de ses composants, tout en minimisant le prix de sa construction. 
%	
%	La modélisation de ce problème prend souvent la forme d'un jeu\cite{goyal14} entre deux participants. L'un construit un réseau, avec ses noeuds et ses arêtes, et en protège certains. Ces actions ont un coût. L'autre participant, l'attaquant, choisit certains noeuds ou arêtes et les supprime s'ils ne sont pas protégés. Dans une modélisation plus fine, les noeuds et arêtes protégés peuvent aussi être supprimés, avec une certaine probabilité [Bravard-Charroin]. Les résultats montrent que si la protection d'un noeud est peu coûteuse par rapport à la création de liens, le réseau optimal est prend la forme d'une étoile dont le centre est protégé.  Au contraire, si la création de liens est moins chère, le réseau optimal sera très dense\cite{goyal14}. Le nombre minimal d'arête pour rendre $k$-connecté un réseau à $n$ noeuds est $\lceil(k * n) / 2\rceil$ selon une démonstration de Frank Harary\cite{harary}.
%
%	
%	Il est nécessaire de classer les différents types de réseaux. Gueye\cite{gueye} introduit des mesures de vulnérabilité d'un réseau en étudiant la connexité de ce réseau après le retrait d'une arête. Une autre topologie\cite{XXX} nous permet d'observer les avantages et inconvéniants de certains réseaux. En particulier, les \emph{scale-free networks}, modèle présnt dans de nombreuses situation, sont efficaces pour propager rapidement des données, mais les n{\oe}uds ayant une connectivité fort, les serveurs, sont assez vulnérables aux attaques. selon ce même article, les réseaux \emph{bimodaux}, dont les n{\oe}uds ont soit $x$, soit $y$ arêtes sortantes sont ceux qui permettent de présenter le meilleur compromis à ce problème.
%
%\section{Problématique retenue}
%
%
%% Réseau de transmission d'information : Architecture, vitesse et fiabilité
%
%%	"""Nous avons cherché à déterminer quelles architectures pemettent de diminuer la vulnérabilité d'un réseau. Pour cela, nous allons étudier deux types d'attaques. """
%
%	Les réseaux sont susceptibles de subir deux types d'attaques. Les unes détruisent des composants du réseau, les autres cherchent à propager de fausses informations. 
%	
%	Quelle architecture choisir pour rendre un réseau le moins vulnérable possible à ces deux types d'attaques ?   
%%
%%\begin{itemize}
%%	\item Quelle architecture du réseau permet la meilleure résistance aux attaques, tout en gardant de bonnes vitesses de propagation des données ?
%%	\item \textbf{Comment rechercher efficacement une information sur un réseau ? Application : Moteur de recherche décentralisé.}
%%	\begin{itemize}
%%		\item Comment s'assurer que les informations reçues sont \emph{de confiance}, c'est-à-dire vraie et correspondant à une réponse à ce qui a été demandé ?
%%	\end{itemize}
%%	
%%	\item (Comment organiser le partage de fichiers efficacement (cas d'un iso \emph{linux}) ? Rajouter des contraintes : combien de n{\oe}uds maximum ?)
%%	
%%\end{itemize}
%
%\section{Objectifs du TIPE}
%
%% Personnel
%
%Notre but consiste à créer un réseau permettant une transmission des informations rapides, tout en résistant efficacement aux attaques, lors desquelles un attaquant prendrait possession de plusieurs n{\oe}uds.
%
%
%
%%	\begin{thebibliography}{56}
%%	
%%%	\bibitem{lamport94}
%%%	  Leslie Lamport,
%%%	  \emph{\LaTeX: a document preparation system},
%%%	  Addison Wesley, Massachusetts,
%%%	  2nd edition,
%%%	  1994.
%%	  
%%	\bibitem{goyal14}
%%	 S. Goyal, A. Vigier
%%	 \emph{Attack, Defense and Contagion in Networks}
%%	 2014-01-18
%%	
%%	\end{thebibliography}
%
%
%\bibliographystyle{plain}
%\bibliography{bib/goyal14} 
%% Goyal ; Harary ; Bravard ; Dziusbynski ; Suto ; Gueye
%% Designing P2P Networks Tolerant to Attacks and Faults Based on Bimodal Degree Distribution,” Journal of Communications, SI on Security and Privacy in Communication Systems and Networks, vol. 7, no. 8, pp.587-595, Aug. 2012.
\end{document}\grid
